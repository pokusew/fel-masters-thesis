\chap[existing-work] Existing Work

Before we started working on our own implementation, we did an extensive search to see what open-source implementations of FIDO2 authenticators were available. The most relevant projects are \ilink[ref:solo]{SoloKeys} and \ilink[ref:opensk]{OpenSK}.


\sec[oss] Open-Source Software and Hardware


\secc[solo] SoloKeys

\begitems

* “the first open source FIDO2 security key”

* v1 launched on Kickstarter in 2018 and v2 in 2021.

* As of today (June 2, 2024), it seems that the project development has stopped.

* v1\urlnote{https://github.com/solokeys/solo1} software written in C for STM32L432.

* v2 software written in Rust for NXP LPC55S69.

\enditems


\secc[opensk] OpenSK by Google

\begitems

* An open-source implementation for security keys written in Rust that supports both FIDO2 and FIDO U2F.

* Actively developed by Google\urlnote{https://github.com/google/OpenSK}.

* Written in Rust, uses Tock OS, the main development target platform is Nordic nRF52840.

\enditems


\secc[nitrokey] Nitrokey

\begitems

* Open Source IT-Security Hardware from a~German company.

* Partially based on the SoloKeys project.

\enditems


\sec[other-relevant] Other Relevant Projects

Two bachelor theses~\cite[borsky_fido2_simulator_2022, kolarik_fido2_keepass_2020] at CTU in Prague (both from CTU FIT).
