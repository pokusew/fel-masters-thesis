\chap[intro] Introduction

Passwords as a means of authentication suffer from many problems. To increase security, multi-factor and passwordless authentication have been introduced. {\sbf FIDO2} is a set of standards based on {\em asymmetric cryptography} that enables easy, secure, and phishing-resistant authentication. It supports {\em passwordless}, {\em second-factor}, and {\em multi-factor} user experiences with embedded (or bound) authenticators (such as biometrics or PINs) or external (or roaming) authenticators (such as FIDO Security Keys, mobile devices, wearables, etc.). All major OSs, browsers, and a growing number of websites and applications support FIDO2 (or the previous more limited standard FIDO U2F).

The goal of this project is to create a new open-source implementation of an external FIDO2 authenticator from scratch that would be well-documented, thoroughly tested, and production-ready. By focusing on the quality and the documentation, this implementation could help others understand the FIDO2 standards by providing a detailed yet accessible insight into the inner workings of these protocols (which the existing implementations lack). In general, this project has the potential to contribute to the popularization of FIDO2 technology.

The result of this project will be an open-source software implementation of a FIDO2 USB hardware external authenticator together with the final project report that will document it. The working of the implementation will be demonstrated on a suitable hardware platform (such as STM32F4). However, the hardware implementation is not the focus of this project.
