\chap[intro] Introduction

Passwords as a means of authentication suffer from many problems. More than 80\% of~confirmed breaches are related to stolen, {\sbf weak, or reused passwords}~\cite[intro_verizon_report_2020]. Password-based credentials are the target of {\sbf phishing attacks}, which are becoming more sophisticated every day. This poses a major threat since 84\% of users reuse the same passwords across multiple sites~\cite[intro_bitwarden_survey_2024]. To limit this threat and to increase the overall security of authentication, two-factor (2FA) or multi-factor (MFA) authentication flows are being increasingly used~\cite[intro_bitwarden_survey_2024].

However, the standard MFA mechanisms, including one-time codes delivered via insecure channels (such as SMS, voice call, or email), TOTPs (e.g., Google Authenticator), and proprietary push notification-based systems, do not provide sufficient security. Not only are they all still susceptible to phishing attacks, but they also greatly hinder the user experience.

To solve this problem, the FIDO Alliance was launched in 2013. It develops and promotes strong authentication standards that “help reduce the world’s over-reliance on passwords”~\cite[fido_alliance_overview].

Its latest set of standards, jointly developed with the W3C (World Wide Web Consortium), is called {\sbf FIDO2}. It is based on asymmetric cryptography, and it enables {\sbf easy, secure, and phishing-resistant authentication} for online services (primarily on the web, but it can be used in native applications as well). It supports {\em passwordless}, {\em second-factor}, and {\em multi-factor} user experiences with {\em platform authenticators} (such as Apple ID with Face ID or Touch ID, Windows Hello, and Google Password Manager on Android) or external {\em (roaming) authenticators} (such as {\sbf FIDO2 security keys}). All~major OSs, browsers, and a growing number of websites and applications support FIDO2~\cite[mdn_webauthn_browser_compatibility, can_i_use_webauthn, passkeys_dev_device_support].

Among the FIDO Alliance's 250 members are all the major technological companies~\cite[fido_alliance_members]. Google (one of the FIDO founding members), Microsoft, and Apple have become vocal advocates for passwordless authentication based on passkeys (the end-user-centric term for FIDO2 credentials) since 2022 when they announced their public commitment to expand support of FIDO2~\cite[fido_alliance_news_2022_05_05].

With the ever-increasing adoption of FIDO2 across the industry~\cite[intro_verizon_report_2020], it is useful to understand~how this technology works and what its benefits are. In order to do that, we decided to~{\sbf create a new open-source implementation of an external (roaming) FIDO2 authenticator from scratch}.
