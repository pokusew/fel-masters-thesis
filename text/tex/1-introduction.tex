\chap[intro] Introduction

Passwords as a means of authentication suffer from many problems. More than 80\% of~confirmed breaches are related to stolen, {\sbf weak, or reused passwords}~\cite[intro_verizon_report_2020]. Password-based credentials are the target of {\sbf phishing attacks}, which are becoming more sophisticated every day. This poses a major threat since 84\% of users reuse the same passwords across multiple sites~\cite[intro_bitwarden_survey_2024]. To limit this threat and to increase the overall security of authentication, two-factor (2FA) or multi-factor (MFA) authentication flows are being increasingly used~\cite[intro_bitwarden_survey_2024].

However, the standard MFA mechanisms, including one-time codes delivered via insecure channels (such as SMS, voice call, or email), TOTPs (e.g., Google Authenticator), and proprietary push notification-based systems, do not provide sufficient security. Not only are they all still susceptible to phishing attacks, but they also greatly hinder the user experience.

To solve this problem, the FIDO Alliance was launched in 2013. It develops and promotes strong authentication standards that “help reduce the world's over-reliance on passwords”~\cite[fido_alliance_overview].

Its latest set of standards, jointly developed with the W3C (World Wide Web Consortium), is called {\sbf FIDO2}. It is based on asymmetric cryptography, and it enables {\sbf easy, secure, and phishing-resistant authentication} for online services (primarily on the web, but it can be used in native applications as well). It supports {\em passwordless}, {\em second-factor}, and {\em multi-factor} user experiences with {\em platform authenticators} (such as Apple ID with Face ID or Touch ID, Windows Hello, and Google Password Manager on Android) or external {\em (roaming) authenticators} (such as {\sbf FIDO2 security keys}). All~major OSs, browsers, and a growing number of websites and applications support FIDO2~\cite[mdn_webauthn_browser_compatibility, can_i_use_webauthn, passkeys_dev_device_support].

Among the FIDO Alliance's 250 members are all the major technological companies~\cite[fido_alliance_members]. Google (one of the FIDO founding members), Microsoft, and Apple have become vocal advocates for passwordless authentication based on passkeys (the end-user-centric term for FIDO2 credentials) since 2022 when they announced their public commitment to expand support of FIDO2~\cite[fido_alliance_news_2022_05_05].

With the ever-increasing adoption of FIDO2 across the industry~\cite[intro_verizon_report_2020], it is useful to understand~how this technology works and what its benefits are. In order to do that, we decided to~{\sbf create a new open-source implementation of an external (roaming) FIDO2 authenticator from scratch}.

\sec[strcuture] Structure

This thesis is structured as follows:
\nl First, in Section~\ref[objective], we define {\sbf the objective} of the thesis.
\nl Second, in Chapter~\ref[fido2], we establish the necessary theoretical foundation by {\sbf describing FIDO2} in detail. We also review existing projects related to our goal in Chapter~\ref[existing-work].
\nl Next, in Chapter~\ref[implementation], we describe {\sbf the actual implementation}, which represents the main contribution of this thesis. Then, we {\sbf test and evaluate} our implementation in Chapter~\ref[evaluation].
\nl Finally, in Chapter~\ref[conclusion], we conclude the thesis by summarizing {\sbf the achieved results}.


\sec[objective] Thesis Objective

The objective of this work is to create a new {\sbf open-source implementation of a FIDO2 USB hardware-based security key} that is fully functional, well-documented and thoroughly tested and offers a~detailed yet accessible insight into the inner workings of FIDO2, which is something that existing open-source implementations currently lack.

In terms of WebAuthn (which we explain in~\ref[fido2]), we aim to create a roaming authenticator with cross-platform attachment using CTAP 2.1 over USB 2.0 (CTAPHID) as the communication protocol, that supports user verification using PIN (CTAP2 ClientPIN), and is capable of storing passkeys (client-side discoverable credentials).

The end result should be similar to proprietary commercial products such as those from Yubico~\ref[img_yubico] or GoTrust~\ref[img_gotrust]. Nevertheless, our work focuses on the software implementation (optimized for use on resource-constrained hardware platforms), not on the physical product or the hardware design. As a~part of our work, we will select a suitable MCU-based hardware platform for our implementation.

We want to create an implementation that is self-contained, with a minimum number of dependencies. By implementing all the key parts from scratch, we can develop a deep understanding of FIDO2/WebAuthn.

Finally, by focusing on the quality and the documentation, this implementation could help others understand the FIDO2 standards, and, in general, contribute to the popularization of FIDO2 technology.

\midinsert
\clabel[img_yubico]{Security Key NFC by Yubico}
\picheight=45mm \cinspic ../images/security-key-nfc-by-yubico.png
\caption/f Security Key NFC by Yubico, one of the most typical FIDO2 USB/NFC security keys (or more precisely, cross-platform roaming authenticators).
\endinsert

\midinsert
\clabel[img_gotrust]{GoTrust Idem Key}
\picheight=45mm \cinspic ../images/gotrust-idem-key.png
\caption/f GoTrust Idem Key, a FIDO2 USB/NFC security key with FIDO Authenticator Certification Level 2 (L2) ~\cite[fido_cert_levels].
\endinsert
