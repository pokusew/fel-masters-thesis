\chap[implementation] Implementation

This chapter describes the main parts of our FIDO2 authenticator implementation.
\_nobreak\_medskip
\_firstnoindent
The complete source code and additional documentation is available on GitHub:\nl
\url{https://github.com/pokusew/lionkey}


\sec Goals

\begitems

* self-contained implementation, with {\sbf a minimum number of dependencies}

* understandable, clear implementation of the standard

* hardware-independent core, making the whole project easily portable

* OS-independent, capable of running as a bare-metal application on the MCU

* testable

* hardware-accelerated cryptography

\enditems


\sec Decisions

We decided to write our implementation in {\sbf C}, which is a very popular language for embedded development. It allows easy access to the underlying hardware (peripheral registers). We did consider using a more high-level language such as Rust, but we wanted to focus primarily on the CTAP2 standard implementation, rather than learning a new language. Moreover, there is already a high-quality CTAP2 implementation in Rust within the \ilink[ref:opensk]{OpenSK} project.

In our implementation, we avoided the use of dynamic memory allocations. All memory is allocated statically or on the stack. We also avoided the use of global variables. All functions only modify the state that they are given. For some parts of the implementation, we also adopted object-oriented techniques to improve testability and enable more flexibility.


\sec Maintaining Quality

Implementing any larger software product requires rigorous quality assurance practices to ensure security, reliability, and maintainability.

We implemented unit tests that allow testing the core of the implementation directly on the host system without the need for the hardware. We tracked code coverage and strived to write better tests. We also configured sanitizers (AddressSanitizer, UndefinedBehaviorSanitizer) to help prevent additional class of possible bugs.

We also set up CI/CD pipeline on GitHub Actions.

Additionally, we relied on the static checks (Clang-Tidy and CLion's advanced static code and flow analyzers). Some of those could be integrated into the CI/CD pipeline as well.

\midinsert
\clabel[img_ci_cd]{CI/CD Setup}
\picw=144mm \cinspic ../images/ci-cd-Screenshot-2025-05-23-at-17.57.46.png
\caption/f A recent run of our CI/CD pipeline in GitHub Actions.
\endinsert

\midinsert
\clabel[img_ctaphid_test]{CTAPHID Test Suite}
\picw=144mm \cinspic ../images/test_ctaphid-Screenshot-2025-05-23-at-18.49.24.png
\caption/f The CTAPHID test suite. 100\% lines coverage (not visible in the screenshot).
\endinsert

\sec Choosing Hardware

We sought a hardware platform that would meet the needs of our implementation while being accessible and usable by the builder community worldwide. Therefore, we established the following criteria:

\begitems
* Available and affordable development kits (boards) with integrated hardware debugger and programmer.
* Freely available and high-quality documentation (Reference Manual, Programming Manual, etc.). No NDA is needed.
* No legacy or deprecated MCUs. Only active products.
* USB 2.0 FS peripheral.
* Single-core 32-bit MCU, such as the Arm Cortex-M line.
* Reasonable amount of RAM and flash memory.
* Hardware-accelerated cryptography (at least ECC on the P-256 curve, AES, SHA-256, True Random Number Generator).
* Nice-to-have: A form factor close to the physical dimensions of a security key.
\enditems

We also came across an interesting presentation from Victor Lomne (NinjaLab) about the hardware used in commercial FIDO(2) security keys~\cite[ninjalab_fido_hw_overview]. For example, Yubico uses NXP and Infineon in its products.

After thoroughly reviewing the available hardware options, we identified two candidates: Nordic Semiconductor nRF52840 and STM32H533.

\ilink[ref:opensk]{OpenSK}, the open-source implementation for security keys written in Rust, has chosen Nordic Semiconductor nRF52840 as its reference hardware target. nRF52840 is a SoC with a 64 MHz Cortex-M4 CPU (ARMv7-M architecture) with FPU, 1 MB Flash, 256 KB RAM, multiprotocol radio support (including Bluetooth), USB 2.0, NFC-A, and Arm TrustZone CryptoCell 310 security subsystem. There are two development boards available: nRF52840 DK (48.10 USD) and nRF52840 Dongle (10.40 USD). The nRF52840 Dongle immediately caught our attention because its form factor almost perfectly corresponds to a standard Yubico-like security key. Thus, we bought samples and evaluated them.

We did not like the quality of the documentation and supporting software libraries compared to STMicroelectronics products, so {\sbf we decided to focus our efforts on our second choice, the STM32H533 MCU on the NUCLEO-H533RE (15.95 USD) development board instead}.

STM32H533 features an Arm Cortex-M33 CPU (ARMv8-M architecture) with TrustZone, FPU, frequency up to 250 MHz, 512 KB of embedded flash memory with ECC, two banks read-while-write, 48-KB per bank with high-cycling capability (100 K cycles) for data flash, 272 KB of SRAM (80-KB SRAM2 with ECC), USB 2.0 FS and hardware-cryptography peripherals (PKA (Public Key Accelerator) with support for ECC P-256, AES, HASH, RNG). Those specs are a great fit for our use case.

\midinsert
\clabel[img_nrf52840]{nRF52840 Dongle}
\picheight=4cm \cinspic ../images/nRF52840-Dongle-rev2-prod.png
\caption/f nRF52840 Dongle
\endinsert

\midinsert
\clabel[img_nucleo_h533re]{NUCLEO-H533RE}
\picheight=6cm \cinspic ../images/NUCLEO-H533RE.png
\caption/f NUCLEO-H533RE
\endinsert


\sec USB

The USB HID and USB layers are implemented using the TinyUSB library, which support the STM32H533's USB peripheral.

There are two endpoints (IN, OUT) with interrupt transfer (64-byte packet max, poll every 5 millisecond). The descriptors (device, config, interface, endpoints, HID report) are set up according to the CTAPHID specification (11.2.8. HID device implementation)~\cite[spec_ctap_2_1].

\midinsert
\clabel[img_lionkey_usb]{LionKey in macOS USB Device Tree Explorer}
\picw=120mm \cinspic ../images/usb-Screenshot-2025-05-19-at-0.41.52.png
\caption/f The following screenshot shows our authenticator in the USB Device Tree Explorer in System Information on macOS 11.

The chosen Vendor ID (VID) and Product ID (VID) come from the \ulink[https://pid.codes/]{pid.codes} project.
pid.codes is a registry of USB PID codes for open source hardware projects. They assign PIDs on any VID they own to any open-source hardware project needing one~\cite[pid_codes].
For the initial version of this thesis, we used their testing VID/PID "0x1209/0x0001". In the future, we might eventually ask for our own PID.

\endinsert


\sec CBOR Parsing

We use TinyCBOR for CBOR parsing.


\sec Arbitrary-length Strings

One challenging aspect of a CTAP2 implementation is the need to support strings of arbitrary length. We implemented an efficient zero-copy solution based on the TinyCBOR's string chunks API. When the strings need to be truncated for storage, the credentials storage submodule takes care of that. When an RP ID gets truncated, we fall back to use its SHA-256 for comparisons.


\sec PIN/UV Auth Protocol

CTAP2 defines two versions of the PIN/UV Auth Protocol. We implemented both of them very elegantly, using OOP techniques.


\sec Cryptography

Our CTAP2 implementation requires an implementation of ECDSA and ECDH on the P-256 curve, AES-256 in CBC mode, SHA-256, HMAC~\cite[rfc5869_hdkf], and HKDF~\cite[rfc5869_hmac].

We developed custom generic implementations of HMAC and HKDF from scratch. They can work with any underlying hash algorithm implementation (ether software-based or hardware-based).

In order to enable testability without the need for special hardware, we first implemented Random Number Generation (allowing custom seed for testing predictability), ECC (ECDSA and ECDH), AES, and SHA-256 using third-party software libraries optimized for use in embedded devices but compatible with standard processor architectures as well. This way we got a baseline that ran on both the host and the target hardware.

Then, we implemented the same set of functions (again using OOP) using the STM32H533 cryptography peripherals RNG, PKA (Public Key Accelerator), SHA, AES. The ability to compare the results between the software- and hardware-based implementation were crucial to debug various tricky issues.


\sec[uart] Debugging Interface via UART

In order to allow easy debugging (and development of the USB communication stack), we implemented a simple bidirectional debugging interface using the UART peripheral. We wrote a simple write syscall implementation so that the printf statements (wrapped in custom "debug/info/error_log" macros) could work as expected. The handling of the incoming debug commands is done in the "app_debug_task()" function.

\_vfil

\sec[persistence] Persistent Storage

There are multiple pieces of data that need to be persisted. Namely, these are:

\begitems

* Whether a PIN has been set. If yes, then also:

\begitems

* Its hash (resp. the first 16 bytes of the 32-byte SHA256 hash) and its length (in Unicode code points).

* The remaining PIN attempts counter, which is initialized to 8. It decreases with each incorrect PIN entry, but it resets back to 8 after the correct PIN is entered. After 3 consecutive incorrect entries (failed attempts), a power cycle (authenticator reboot/reconnect) is required by the spec. This is done so that malicious applications running on the platform cannot block the device without user interaction~\cite[spec_ctap_2_1]. It is necessary because the authenticatorClientPIN subcommands used to enter the PIN can be invoked by applications without any user interaction (user presence check).

\enditems

* The current minimum PIN length (in Unicode code points). We store this so that we easily add support for the optional "setMinPINLength" subCommand of the "authenticatorConfig" command and the Minimum PIN Length Extension (minPinLength) in the future).

* Credentials

\begitems

* It is absolutely necessary to store all client-side discoverable credentials (passkeys).

* But the non-discoverable credentials do not need to be stored. Instead, we could generate (random) and store a single symmetric encryption key and use it to encrypt (wrap) the private keys of non-discoverable credentials as we described in the \ilink[ref:credential_storage_modality]{Credential Storage Modality} section.

* However, in our current implementation, we store all credentials.

\enditems

\enditems

As of writing this thesis, the state persistence is the last remaining piece of functionality that is still a work in progress.
