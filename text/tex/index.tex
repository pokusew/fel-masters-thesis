% CTUstyle3 template
\input ctustyle3

\worktype [M/EN]

\faculty {F3}
\department {Department of Computer Science}

\title {FIDO2 USB Security Key}
\titleCZ {\nl FIDO2 USB bezpečnostní klíč}
\author {Martin Endler}
\date {May 2025}

\supervisor {Ing. Jan Sobotka, Ph.D.}
\studyinfo  {Open Informatics – Cybersecurity}
\workinfo   {\url{https://github.com/pokusew/fel-masters-thesis}}

% The two-page KOS-generated Assigment goes after the Title Page
\specification {
	\vbox to0pt{\vskip-32mm\centerline{\inspic ../../formalities/Thesis_Assignment_Martin_Endler_FIDO2_USB_Security_Key_v5_p1.pdf }\vss}
	\vfil\break
	\vbox to0pt{\vskip-32mm\centerline{\inspic ../../formalities/Thesis_Assignment_Martin_Endler_FIDO2_USB_Security_Key_v5_p2.pdf }\vss}
}
% After that goes the one-page KOS-generated Declaration PDF
\declaration {
	\vbox to0pt{\vskip-32mm\centerline{\inspic ../../formalities/Thesis_Declaration.pdf }\vss}
}

\abstractEN {

	FIDO2 is a~set of standards based on asymmetric cryptography that enables easy, secure, and phishing-resistant authentication. The strong support from Google, Microsoft, Apple, and other major technology companies in the FIDO Alliance is driving the adoption of FIDO2 across the industry.

	This thesis focuses on creating a~new open-source FIDO2 USB hardware external authenticator. We provide an overview of the relevant technologies and standards, specifically Web Authentication and CTAP2. We also review existing similar projects. Then, we proceed to the actual implementation. We document our decisions and the most important parts. The resulting fully functional CTAP 2.1 compliant implementation is publicly available on GitHub. It utilizes hardware-accelerated cryptography on the STM32H533 MCU, passes FIDO2 conformance tests, and is usable for authentication on real WebAuthn-enabled websites.

	We hope our work will contribute to the popularization and better understanding of the FIDO2 technology.

}
\keywordsEN {
	FIDO2, WebAuthn, CTAP, CTAP2, USB, CTAPHID, security key, passkeys, authenticator, passwordless authentication, multi-factor authentication, MFA, second-factor authentication, 2FA, asymmetric cryptography, public key cryptography, Elliptic-curve cryptography, ECC
}

\abstractCZ {

	FIDO2 je sada standardů založených na asymetrické kryptografii, která umožňuje snadnou, bezpečnou a~proti phishingu odolnou autentizaci. Silná podpora ze strany společností Google, Microsoft, Apple a~dalších významných technologických firem v~rámci FIDO \nobreak{Alliance} podporuje adopci FIDO2 v~celém odvětví.

	Tato práce se zaměřuje na vytvoření nového open-source FIDO2 USB hardwarového externího autentizátoru. Nejdříve popisuje relevantní technologie a~standardy, zejména Web Authentication a~CTAP2. Také zmiňuje existující podobné projekty. Poté přistupuje k~samotné implementaci. Dokumentuje rozhodnutí a~nejdůležitější části. Výsledná plně funkční implementace splňující CTAP 2.1 specifikaci je veřejně dostupná na GitHubu. Využívá hardwarově akcelerovanou kryptografii na STM32H533, prochází testy shody FIDO2 a~je použitelná pro autentizaci na skutečných webových stránkách s~podporou WebAuthn.

	Doufáme, že naše práce přispěje k~popularizaci a~lepšímu porozumění technologii FIDO2.

}
\keywordsCZ {
	FIDO2, WebAuthn, CTAP, CTAP2, USB, CTAPHID, bezpečnostní klíč, passkeys, autentizátor, přihlášení bez hesla, vícefaktorová autentizace, MFA, dvoufaktorová autentizace, 2FA, asymetrická kryptografie, kryptografie s~veřejným klíčem, kryptografie nad eliptickými křivkami, ECC
}

\thanks {

	First, I would like to thank my~supervisor Ing.~Jan~\nobreak{Sobotka},~Ph.D.
	for his guidance, support, and patience. Second, I would like to thank my family and close friends
	for always supporting me throughout my studies.

}

% link glossary data
\input glossary

%%%%% <--   % The place for your own macros is here.

%\draft     % Uncomment this if the version of your document is working only.
%\linespacing=1.7  % uncomment this if you need more spaces between lines
% Warning: this works only when \draft is activated!
%\savetoner        % Turns off the lightBlue backround of tables and
% verbatims, only for \draft version.
%\blackwhite       % Use this if you need really Black+White thesis.
%\onesideprinting  % Use this if you really don't use duplex printing.

% make title page, acknowledgment, contents etc.
\makefront

% contents
\input 1-introduction
\input 2-fido2
\input 3-existing-work
\input 4-implementation
\input 7-conclusion

% appendices
\input appendices

% document end
\bye
