\chap[conclusion] Conclusion

In this thesis, {\sbf we successfully implemented} a fully functional FIDO2 USB hardware external authenticator (also called FIDO2 USB security key).

First, we outlined the objective of our work and its motivation. Then, we established the necessary theoretical foundation by describing the Web Authentication and CTAP2 standards, which are together known as FIDO2. We also reviewed existing projects related to our goal.

Then, we proceeded to the actual implementation, which represents the main contribution of~this thesis. We documented our decisions and the most important parts of the implementation. All code is versioned using Git and is publicly {\sbf available online on GitHub\urlnote{https://github.com/pokusew/lionkey}}. The repository includes build and run instructions. We employed software development best practices, including unit testing and CI/CD pipeline.

Finally, {\sbf we tested} our authenticator {\sbf with real websites} with different configurations. In all cases, {\sbf it worked great}, including with Google Account and GitHub.
Furthermore, we tested our authenticator using the official FIDO Conformance Tools testing application. Our implementation passed the entire CTAP 2.1 test suite, demonstrating the quality of our work.

To support adoption of our implementation by people from all over the world, we gave our implementation a name – LionKey – and we designed a logo. We also purchased the domain "lionkey.dev" with the aim to launch a website with more detailed documentation.

\midinsert
\picheight=45mm \cinspic ../images/lionkey-logo.pdf
\caption/f The LionKey logo.
\endinsert
