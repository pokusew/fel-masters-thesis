\chap[fido2] FIDO2

In this chapter, we provide a~more detailed description of {\sbf FIDO2}.

\_vskip 2mm

FIDO2 is a~set of related specifications, jointly developed by the FIDO Alliance and the W3C (World Wide Web Consortium), that together enable {\sbf easy, secure, and phishing-resistant authentication} for online services (primarily on the web, but they can be used in native applications as well).

\_vskip 2mm

\_firstnoindent
{\sbf The specifications are:}

\begitems

* {\sbf Web Authentication (WebAuthn) API} by World Wide Web Consortium (W3C)

\_vskip 1mm
\_firstnoindent
This is the core specification that {\sbf defines and describes all the key concepts}, some of which we cover in the following sections. More specifically, it “defines an API enabling the creation and use of strong, attested, scoped, public key-based credentials by web applications, for the purpose of strongly authenticating users”~\cite[spec_webauthn_level_2].

\_vskip 1mm
\_firstnoindent
{\sbf Level 2}~\cite[spec_webauthn_level_2] (W3C Recommendation from April 8, 2021) is the latest stable version.

\_vskip 1mm
\_firstnoindent
{\sbf Level 3}~\cite[spec_webauthn_level_3] is being actively developed and some of its new features (hybrid transport, conditional mediation, hints, credentials backup) are already supported by browsers and authenticators.

\_vskip 2mm
* {\sbf Client to Authenticator Protocol (CTAP)} by FIDO Alliance

\_vskip 1mm
\_firstnoindent
“This specification describes an application layer protocol for {\sbf communication between a~roaming authenticator and another client}/platform, as well as bindings of this application protocol to different transport protocols”~\cite[spec_ctap_2_1].

\_vskip 1mm
\_firstnoindent
{\sbf CTAP 2.1}~\cite[spec_ctap_2_1] (Proposed Standard from June 21, 2022) is the previous version.

\_vskip 1mm
\_firstnoindent
{\sbf CTAP 2.2}~\cite[spec_ctap_2_2] (Proposed Standard from February 28, 2025) is the latest stable version. The main difference compared to CTAP 2.1 is the addition of hybrid transports. Apart from that and a~few new extensions, it is completely the same as CTAP 2.1 (There is no new version identifier for CTAP 2.2, authenticators still use {\tt FIDO\_2\_1}). Because none of those changes are relevant for our work, we will refer to CTAP 2.1 for the rest of our text.

\_vskip 2mm
* There is also FIDO U2F (now referred to as CTAP1 or CTAP1/U2F), which is a~predecessor of FIDO2 that can be used only for two-factor authentication (as a~second factor). FIDO2 authenticators can be backwards-compatible with CTAP1 (FIDO U2F). To avoid unnecessary complexity, we will not add support for CTAP1/U2F to our implementation. Therefore, we will not describe it.

\enditems

\_vfill

\sec[pka] Public Key Cryptography


FIDO2 relies on public key cryptography (also called asymmetric cryptography)~\cite[spec_webauthn_level_2].

{\em Public key cryptography} uses the concept of a~key pair. Each key pair consists of a~{\sbf private key} (which must be kept secret) and a~corresponding {\sbf public key} (which can be openly distributed without compromising security). The public and private keys are mathematically related. They are generated with cryptographic algorithms based on mathematical problems termed {\em one-way functions}~\cite[wiki_pka]. It is not possible to calculate the private key from the public key. One of the most common public key cryptosystems is RSA that relies on the difficulty of factoring the product of two large prime numbers~\cite[wiki_rsa]. Another widely used public key cryptosystem is Elliptic Curve Cryptography (ECC), which is based the algebraic structure of elliptic curves over finite fields, specifically the difficulty of the Elliptic Curve Discrete Logarithm Problem (ECDLP)~\cite[wiki_ecc].

\sec[func_desc] Functional Description

The core idea is that FIDO2/WebAuthn allows servers (i.e., websites, referred to as {\sbf Relying Parties}, {\sbf RP}) to register and authenticate users using public key cryptography.

\midinsert
\clabel[webuathn_flow_auth_simple]{Simplified WebAuthn {\nobreak Authentication} Flow}
\picw=15cm \cinspic ../images/simple-schema.pdf
\caption/f Simplified WebAuthn authentication flow. The credentials belong to the user and are managed by an authenticator, with which the Relying Party interacts through the client platform.
\endinsert

Instead of a~password, a~private-public keypair (known as a {\sbf credential}) is created for a~website during registration. While the {\sbf private key} is stored securely on the user's device (inside the {\sbf authenticator}) and it never leaves it, the {\sbf public key} and an authenticator-generated credential ID are sent to the server where they are stored.

Then, during {\sbf authentication}, the RP's server then uses that {\sbf public key} to verify the user's identity by verifying the user's possession of the corresponding private key. More specifically, the server sends the credential ID and a~random data called a~challenge to the client (the user's browser). The client bounds the challenge to the RP ID by hashing the challenge together with the RP ID and producing clientDataHash. The client passes RP ID, credential ID, and clientDataHash to the user's authenticator. The authenticator looks up the {\sbf private key} by the RP ID and credential ID and it signs the concatenation of authenticatorData (which contains info about the authenticator) and clientDataHash using the {\sbf private key}. It passes this signature back to the server via the client. The server checks that all values in the clientDataJSON and in the authenticatorData have the expected values (especially RP ID). Then it computes the expected hash using the clientDataJSON and authenticatorData. Finally, it uses the {\sbf public key} to verify the signature.

The exact signature algorithm depends on the supported algorithms by the RP and by the authenticator. WebAuthn does not list the specific algorithms, instead it refers to the IANA CBOR Object Signing and Encryption (COSE) Algorithms Registry\urlnote{https://www.iana.org/assignments/cose/cose.xhtml\#algorithms}. One of the common algorithm identifiers is "-7", which stands for Elliptic Curve Digital Signature Algorithm (ECDSA) on the P-256 curve (also called secp256r1 or prime256v1) with SHA-256 as the hashing function.


\sec Key Benefits

Based on the description in the previous section, we can clearly see some of the security and privacy benefits of FIDO2 authentication.

\begitems

* {\sbf Security}

\begitems

* A~new unique public-private key pair (credential) is created for every relying party (and every authenticator the user uses).

* All credentials are scoped to the relying party (website).

* The private key never leaves the authenticator.

* The RP's server does not store any secrets.

* The possession of the private key is proved by signing a~random challenge generated by the RP.

* This security model {\sbf eliminates all risks of phishing, all forms of password theft} (there is {\sbf no shared secret to steal}) and {\sbf replay attacks} (thanks to the random challenge).

\enditems

* {\sbf Privacy}

\begitems

* The whole FIDO2 is designed with privacy in mind. All credentials are unique and they are scoped to the relying party (website). They cannot be used to track users in any way.

* Plus, biometric data, when used for user verification, never leaves the user's device (authenticator).

\enditems

* {\sbf User Experience}

\begitems

* Users interact with the authenticator by (typically) simply touching it to provide consent with registration and authentication.

* Credentials can be further protected and their use might require local user verification, which, based on the authenticator capabilities, might utilize PIN or biometric methods.

* Users can use cross-platform {\em roaming authenticators} such as FIDO2 security keys (which communicate via USB, NFC, Bluetooth) or they might rely on their-device built-in {\em platform authenticators} (such as Apple ID with Face ID or Touch ID, Windows Hello, and Google Password Manager on Android) that are part of the client device.

\enditems

\enditems


\sec Terminology

In the previous sections, we briefly introduced the key parties involved in FIDO2 – {\em the relying party (RP), the client, and the authenticator}. In this section, we provide more detailed information about them and their behavior. We also define additional related terms to aid in our explanations.


\secc[def_rp] Relying party (RP)

“An entity whose application utilizes the WebAuthn API to register and authenticate users, and which stores the public key.”~\cite[spec_webauthn_level_2] Typically, such an application has both client-side code to invoke the WebAuthn API on the client (typically in the browser) and server-side code to validate responses and store details about registered credentials (their IDs and public keys) in some sort of database.

\secc[def_client] Client

An entity (typically a~web browser or a~similar application) that acts as an intermediary between the relying party and the authenticator~\cite[spec_webauthn_level_2].


\secc[def_client_device] Client Device

“The hardware device on which the client runs, for example a~smartphone, a~laptop computer or a~desktop computer, and the operating system running on that hardware.”~\cite[spec_webauthn_level_2]


\secc[def_client_platform] Client Platform

“A client device and a~client together make up a~client platform. A~single hardware device may be part of multiple distinct client platforms at different times by running different operating systems and/or clients.”~\cite[spec_webauthn_level_2]


\secc[def_authenticator] Authenticator

“A cryptographic entity, existing in hardware or software, that can register a~user with a~given Relying Party” (i.e., generate and store a~public-private key pair with appropriate metadata) and “later assert possession of the private key during authentication”.~\cite[spec_webauthn_level_2] The private keys never leave the authenticator.

The authenticators that are part of the client device are referred to as {\em platform authenticators} (such as Apple ID with Face ID or Touch ID, Windows Hello, and Google Password Manager on Android), while those that are reachable via cross-platform transport protocols (USB, NFC, Bluetooth) are referred to as {\em roaming authenticators} (such as {\sbf FIDO2 USB security keys}).

\secc[def_aaguid] AAGUID

Authenticator Attestation Globally Unique Identifier. A~128-bit identifier indicating the type (e.g.\ make and model) of the authenticator, chosen (randomly generated) by its manufacturer.

\secc[def_credential] Credential

A~public-private key pair that is used for authentication.
It is bound to an RP ID.
A~credential is identified by a {\sbf Credential ID}, which is an opaque probabilistically-unique byte string generated by authenticator that can be up to 1023 bytes long.
It is associated with a~user account, which is identified by a {\sbf user handle} ("user.id"). A~user handle is an opaque byte sequence with a~maximum size of 64 bytes, generated by the RP and stored together with the credential by the authenticator. Optionally, the user account information within the credential can also contain "name" and "displayName", which are both arbitrary-length strings.
Every credential is either discoverable (also called client-side discoverable) or non-discoverable. RP controls which type of credential will be created during registration by passing an appropriate option to the WebAuthn API. Discoverable credentials are also called {\sbf“passkeys”}~\cite[spec_webauthn_level_3].
RP pass credential IDs during registration (in the excludeList) to prevent users from creating multiple credentials for the same user account on a~single authenticator.
RP can also pass credential IDs during authentication (in the allowList) to explicitly allow authentication (i.e., generating an assertion, resp. a~signature) only using those credentials. When a~credential is non-discoverable, its Credential ID must be passed to authenticator during each authentication call.


\secc[def_uv] User Verification

“The process by which an authenticator locally authorizes the invocation of an operation. Various authorization gesture modalities are possible, for example, a~touch plus pin code, password entry, or biometric recognition (e.g., presenting a~fingerprint). The intent is to distinguish individual users.”~\cite[spec_webauthn_level_2].

The “user verification does not give the RP a~concrete identification of the user, but when two or more ceremonies with user verification have been done, it expresses that it was the same user that performed all of them. The same user might not always be the same natural person, however, if multiple natural persons share access to the same authenticator.”~\cite[spec_webauthn_level_2].


\secc[def_up] User Presence

“A test of user presence is a~simple form of authorization gesture and technical process where a~user interacts with an authenticator by (typically) simply touching it.”~\cite[spec_webauthn_level_2].


\secc[credential_storage_modality] Credential Storage Modality

Authenticators (such as FIDO2 hardware security keys) might have a~limited storage capacity. Until now, we have assumed that {\sbf the private keys} must be stored in the authenticator. In fact, the WebAuthn allows {\sbf two different storage strategies}:

\begitems \style n

* The private key is stored {\sbf in persistent storage} embedded in the authenticator. This strategy is necessary for {\sbf discoverable credentials} (also called {\sbf client-side discoverable credentials} or {\sbf passkeys}) that can be used as a~first factor.

* The private key is {\sbf stored within the credential ID}. Specifically, the private key is encrypted {\sbf (wrapped)} such that only the authenticator can decrypt (i.e., unwrap) it. The resulting ciphertext is {\sbf the credential ID}.

Typically, the authenticator uses symmetric AES encryption, where the symmetric encrypt/decrypt key (sometimes called master key) is securely stored in the authenticator and never leaves it. Therefore, this strategy, if correctly implemented, is as secure as the first one.

Note that this strategy is possible because {\sbf the RP passes the credential ID} to the authenticator {\sbf during authentication}. This is the case for the non-discoverable credentials. Therefore, the authenticator does not need to store any information about such a~credential in its persistent storage. This way it can support virtually an unlimited number of credentials. However, this strategy {\sbf cannot} be used for discoverable credentials, where the RP does {\sbf not} pass the credential ID to the authenticator and instead the authenticator must be able to list all existing credentials for that RP.

\enditems

Note that the authenticators {\sbf might support both strategies} and use different storage strategies for different credentials (typically only use the persistent storage when it is strictly necessary, i.e., for {\em discoverable} credentials).

For more information, see 6.2.2. Credential Storage Modality in~\cite[spec_webauthn_level_2].


\sec Ceremonies

In this section, we build on the information provided in the \ilink[ref:func_desc]{Functional Description} section, and we cover the {\sbf registration} and {\sbf authentication} flows in more detail. These flows are referred to as “ceremonies” by the WebAuthn specification because they extend the concept of a~computer {\em communication protocol} with human-computer interactions.

In the following descriptions, we assume {\sbf a~standard web application} that runs client-side code to invoke the WebAuthn JavaScript APIs in the browser and to communicate with its server, which validates responses and stores/retrieves details about registered credentials (their IDs and public keys). In this case, the browser represents the \nobreak{WebAuthn} client. Note that WebAuthn can also be used in native applications where platform-specific vendor-provided APIs are used in place of the WebAuthn JavaScript APIs. On Android, these are FIDO2 API for Android\urlnote{https://developers.google.com/identity/fido/android/native-apps} and/or Credential Manager API\urlnote{https://developers.google.com/identity/android-credential-manager}. iOS provides similar APIs\urlnote{https://developer.apple.com/documentation/authenticationservices/public-private-key-authentication}.


\secc Registration

During registration, a~new credential is created on an authenticator and registered with a~Relying Party server. Registration must happen before a~user can use their authenticator to authenticate.

The following diagram~\ref[webuathn_flow_reg] from depicts the registration flow:

\midinsert
\clabel[webuathn_flow_reg]{WebAuthn Registration Flow}
\picw=12cm \cinspic ../images/webauthn-flow-reg.pdf
\caption/f The WebAuthn registration flow.
\endinsert

The user visits a~website, for example, "example.com", which serves up a~script. At this point, the user may already be logged in using a~legacy username and password, or additional authenticator, or other means acceptable to the Relying Party. Or the user may be in the process of creating a~new account.

\begitems \style n \itemnum -1

* The user initiates the registration process by interacting with the application in the browser. The application sends a~request to the server to start the registration.

* The server responds with "PublicKeyCredentialCreationOptions" that contains information about the user (the opaque {\em user handle}, also called "user.id"), preferred authenticators (roaming vs. platform) and {\em supported algorithms}. It also includes a~random data called {\sbf challenge}. This is to prevent replay attacks.

* The client-side code in the browser invokes the WebAuthn API method \nl "navigator.credentials.create()" and passes the options from the server. The browser {\sbf appends the RP ID} (the website origin, i.e., example.com). This is necessary for the {\sbf correct scoping} of the credentials to the RP (every credential is  {\em bound} to an RP ID). Also, the browser appends {\sbf clientDataHash}, which is a~hash over the operation type ("webauthn.create" or "webauthn.get"), the challenge, and the origin. Then, the browser locates a~suitable authenticator, establishes a~connection to it, and sends the "authenticatorMakeCredential" request. The communication between the browser (the client) and the authenticator uses CTAP2 in case of roaming authenticators (e.g., a~FIDO USB security key) or some platform-specific communication channel in case of platform authenticators (e.g. Windows Hello).

* The authenticator asks the user for some sort of authorization gesture to provide consent. It may involve user verification (PIN entry or biometric check) or it may only be a~simple test of user presence. If the user authorizes the registration, the authenticator {\sbf generates a~new credential} (a public-private key pair and a~credential ID). The private key is securely stored in the authenticator and never leaves it.

* The authenticator takes {\sbf the generated public key} and {\sbf the credential ID}, appends the clientDataHash and information about itself and signs all this data with either its attestation private key or with the credential private key (so called self-attestation).

* The client-side code in the browser obtains the response from the authenticator (as the result of the "navigator.credentials.create()" call) and sends it to the server.

* The server verifies the data. Specifically, it checks the attestation signature. This way, the server ensures that all the credential creation options are respected and that the response is indeed related to the initial request challenge. If everything matches, {\sbf the server saves the public key and the credential ID to the database} and associates them with the user.

\enditems

\_vfil


\secc Authentication

This is the ceremony when a~user with an already registered credential visits a~website and wants to authenticate using the credential ("authenticatorGetAssertion" request). During authentication, the server uses the previously registered public key to verify the user's identity. Specifically, it verifies the user's possession of the private key by validating the authenticator-generated assertion signature. The complete authentication flow is depicted in Figure~\ref[webuathn_flow_auth].

% TODO: Describe the authentication flow here in the same detail as the registration flow above.
Since we have already described the authentication in sufficient detail in the \ilink[ref:func_desc]{Functional Description} section, we omit the detailed description here and ask the reader to refer to the corresponding description there.

\midinsert
\clabel[webuathn_flow_auth]{WebAuthn Authnentication Flow}
\picw=12cm \cinspic ../images/webauthn-flow-auth.pdf
\caption/f The WebAuthn authentication flow.
\endinsert


\sec CTAP2

WebAuthn defines the key concepts and principles of the whole FIDO2 functionality. However, it does not define the communication protocol between the client and the authenticator. It introduces an abstract functional model for a~WebAuthn Authenticator. This model defines the necessary observable behavior. Specifically, it defines the authenticator data\textit{ }structure, signature formats (attestations and assertions), and the authenticatorMakeCredential, authenticatorGetAssertion, and authenticatorCancel operations using an abstract notion of connection and session between the client and the authenticator.

CTAP2 represents a~concrete instantiation of this model. It defines the actual communication protocol and its bindings to various transport protocols. The specification defines multiple abstraction levels.


\secc Authenticator API

At the highest level of abstraction, CTAP2 defines a~conceptual API consisting of operations (also called commands or requests). Each command accepts input parameters and returns either a~response with data on success or an error code. The returned response depends on the parameters and the authenticator’s internal state. However, there are also special commands called stateful commands, which do not accept any parameters but require an additional command state to be maintained. “Each such command uses and updates a~state that is initialized by a~corresponding state initializing command.”~\cite[spec_ctap_2_1]

Here is a~list of all mandatory CTAP 2.1 commands with their command codes. Note that some of the commands have also subcommands (technically, the subcommand code and the subcommand parameters are parameters of the parent command).

\begitems

* {\sbf authenticatorGetInfo (0x04)} – “Using this method, clients can request that the authenticator report a~list of its supported protocol versions and extensions, its AAGUID, and other aspects of its overall capabilities. Clients should use this information to tailor their command parameters choices.”~\cite[spec_ctap_2_1]

* {\sbf authenticatorClientPIN (0x06)} – See \ilink[ref:pin_protcol]{PIN/UV Auth Protocol}

* {\sbf authenticatorMakeCredential (0x01)} – “request generation of a~new credential in the authenticator”~\cite[spec_ctap_2_1]

* {\sbf authenticatorGetAssertion (0x02)} – “request cryptographic proof of user authentication as well as user consent to a~given transaction, using a~previously generated credential that is bound to the authenticator and RP ID”~\cite[spec_ctap_2_1]

* {\sbf authenticatorGetNextAssertion (0x08)} – (stateful command ) “The client calls this method when the authenticatorGetAssertion response contains the numberOfCredentials member and the number of credentials exceeds 1.”~\cite[spec_ctap_2_1]

* {\sbf authenticatorCredentialManagement (0x0A)} – “This command is used by clients to manage discoverable credentials on the authenticator.”~\cite[spec_ctap_2_1]

* {\sbf authenticatorSelection (0x0B)} – “This command allows the client to let a~user select a~certain authenticator by asking for user presence.”~\cite[spec_ctap_2_1]

* {\sbf authenticatorReset (0x07)} – “Invalidates all generated credentials, erases all discoverable credentials, resets PIN”~\cite[spec_ctap_2_1]

\enditems

\_vfil

\secc[pin_protcol] PIN/UV Auth Protocol

CTAP2 offers two different ways to perform user verification:

\begitems

* {\sbf ClientPIN} – The entry of the PIN is mediated by the client. The user enters the PIN into the client’s UI (e.g., a~browser dialog), and the client sends it encrypted to the authenticator. ClientPIN is useful when the authenticator does not have its input UI (e.g., it is a~security key without any display or keypad).

* {\sbf Built-in User Verification} – User verification is performed entirely within the authentication boundary, e.g., the authenticator has a~fingerprint sensor or a~secure built-in input UI (e.g., built-in keypad).

\enditems

The PIN/UV auth protocol (aka pinUvAuthProtocol) handles both ways of user verification. The ClientPINs are encrypted when sent to an authenticator and exchanged for a~pinUvAuthToken to authenticate subsequent commands. “Authenticators supporting built-in user verification methods can provide a~pinUvAuthToken (again, to authenticate subsequent commands) upon performing the built-in user verification. The pinUvAuthToken is a~randomly-generated, opaque bytestring that is large enough to be effectively unguessable.”~\cite[spec_ctap_2_1] Both PIN and pinUvAuthToken are never sent in plaintext. Instead, they are encrypted using a~shared secret that is obtained by performing an ECDH key agreement. However, the plain ECDH key agreement (as is any unauthenticated DH key agreement) is vulnerable to MitM attacks (e.g., an active attacker with the ability to modify USB packets). % TODO: Those weaknesses are studied in (TODO: Add references).

The authenticatorClientPIN (0x06) command, resp. its eight subcommands, implement the necessary operations to get the number of remaining PIN/UV retries, obtain a~shared secret (using ECDH), set a~PIN, change existing PIN, exchange PIN for a~pinUvAuthToken, and perform built-in user-verification to get a~pinUvAuthToken.

Furthermore, PIN/UV Auth Protocol provides an internal API, which is used by the rest of the Authenticator API commands to validate the pinUvAuthParam. The pinUvAuthParam is used to convey the user verification status while also authenticating the parameters of commands such as authenticatorMakeCredential, authenticatorGetAssertion, and authenticatorCredentialManagement. The client uses the previously obtained pinUvAuthToken to compute an HMAC over all or some of the command parameters. The authenticator verifies the HMAC by computing it from the relevant received parameters and the current pinUvAuthToken (if in use). A~successful verification indicates that a~user verification performed in the recent authneticatorClientPIN command can now be used to perform the requested operation (makeCredential, getAssertion, or credentialManagement).


\secc Message Encoding

CTAP2 encodes all command parameters and response data using Concise Binary Object Representation (CBOR). This encoding was chosen because many transports (for example, NFC) “are bandwidth-constrained, and serialization formats such as JSON are too heavy-weight for such environments”~\cite[spec_ctap_2_1]. Furthermore, CTAP2 introduces the CTAP2 canonical CBOR encoding form, which further restricts the CBOR format with additional encoding rules (such as NO indefinite-length arrays and maps, integers MUST be encoded as small as possible, etc.).

Each operation in the conceptual Authenticator API has a~command code (1 byte) assigned. The request message is constructed by prepending the command code to the CBOR-encoded parameters. Analogically, the response message is constructed by prepending the error code (0 for success) to the CBOR-encoded response data. Since error responses do not have any data, they are only one byte (the error code).


\secc CTAPHID

One of the defined transport protocols for CTAP2 is USB, resp. USB HID. The protocol is called CTAPHID, and it implements logical channel multiplexing so that multiple clients (browsers, apps, OS) on the same client device can communicate with the authenticator concurrently. Next, it introduced the concept of a~transaction. The CTAPHID transaction consists of a~CTAPHID request followed by a~CTAPHID response message. The transactions are always initiated by clients. Request and response messages are divided into individual fragments, known as packets. Packets are directly mapped onto USB HID reports. The fragmentation and channel multiplexing add some overhead to the raw CTAP2 messages (11\% – 8\% based on the payload length when the HID report size is 64 bytes).

The USB HID report descriptor plays a~key role in the automatic \textit{device discovery}. The CTAPHID protocol is designed with the objective of driver-less installation on all major host platforms. Browsers and other clients use the standard USB HID driver included within the OS. The HID \textit{Usage Page} field within the authenticator’s HID report descriptor is set to the value "0xF1D0", which is registered to the FIDO Alliance (see HID Usage Tables 1.6~\cite[hid_usage_tables_1_6]).


% \sec USB
% \sec USB HID
