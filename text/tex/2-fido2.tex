\chap[fido2] FIDO2

{\sbf FIDO2} is a set of standards based on {\em asymmetric cryptography} that enables easy, secure, and phishing-resistant authentication. It supports {\em passwordless}, {\em second-factor}, and {\em multi-factor} user experiences with embedded (or bound) authenticators (such as biometrics or PINs) or external (or roaming) authenticators (such as FIDO Security Keys, mobile devices, wearables, etc.).

{\sbf The specifications are:}

\begitems

* \ilink[cite:spec_ctap_2_1]{Client to Authenticator Protocol (CTAP) by FIDO Alliance}

* \ilink[cite:spec_webauthn_level_2]{Web Authentication (WebAuthn) API by World Wide Web Consortium (W3C)}

\enditems

Note: FIDO U2F is a predecessor of FIDO2 that can be used only for two-factor authentication (as a second factor).


\sec Functional Description

The basic idea is that the credentials belong to the user and are managed by a WebAuthn/FIDO2 Authenticator, with which the WebAuthn Relying Party interacts through the client platform. Relying Party scripts can (with the user’s consent) request the browser to create a new credential for future use by the Relying Party~\cite[spec_webauthn_level_2].


\secc[def_rp] Relying party (RP)

An entity whose application utilizes the WebAuthn API to register and authenticate users, and which stores the public key.


\secc[def_authenticator] Authenticator

A cryptographic entity that handles generating and storing keys and performing cryptographic operations.

The private keys never leave the authenticator.


\seccc Credential Storage Modality

The keys might be either stored in persistent storage embedded in the authenticator (necessary for client-discoverable/resident credentials that can be used as a first factor), or they might be derived from the credential ID (then the authenticator can support virtually an unlimited number of credentials).

For more information, see 6.2.2. Credential Storage Modality in~\cite[spec_webauthn_level_2].


\secc[def_client] Client

An entity that acts as an intermediary between the \ilink[ref:def_rp]{relying party} and the \ilink[ref:def_authenticator]{authenticator} (typically a web browser or a similar application).


\secc[def_client_device] Client Device

The hardware device on which the WebAuthn \ilink[ref:def_rp]{Client} runs, for example a smartphone, a laptop computer or a desktop computer, and the operating system running on that hardware.


\secc[def_client_platform] Client Platform

A client device and a client together make up a client platform. A single hardware device may be part of multiple distinct client platforms at different times by running different operating systems and/or clients.
